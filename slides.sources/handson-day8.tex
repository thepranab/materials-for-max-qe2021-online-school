\documentclass[landscape]{foils}
\usepackage[pdftex]{color}
\usepackage{url}
\usepackage[pdftex]{graphicx}
\usepackage{eso-pic}
\usepackage[top=2cm, bottom=2cm, outer=0cm, inner=0cm]{geometry}
\usepackage{listings}
\usepackage{amsmath}

\input{aliases}

\begin{document}
\AddToShipoutPictureBG*{\includegraphics[width=\paperwidth,height=\paperheight]{figs/qe2021-background-4x3.png}}

\blue
%\includegraphics[width=1.0\textwidth]{figs/QE2019-logo.pdf}
~\\
\vspace*{4cm}
\MyLogo{~}
\vspace{5em}
\begin{center}
  {\burgundy\LARGE\bf QE-2021: Hands-on session -- Day-8}\\[2em]
  {\burgundy\LARGE (Ab Initio Molecular Dynamics)}
  ~\\[1.5em]
  %\large Pietro Delugas, Anton Kokalj, Paolo Giannozzi, Xiang Mei Duan,\\
  %Matic Pober\v{z}nik, Unmesh Mondal, Matej Hu\v{s}, Yaning Cui
\end{center}

%%%%%%%%%%%%%%%%%%%%%%%%%%%%%%%%%%%%%%%%%%%%%%%%%%%%%%%%%%%%% 
\Head{QE-2021: Hands-on session -- Day-8}
\MyLogo{\burgundy {\bf QE-2021}: MaX School on Advanced Materials and Molecular Modelling}
\rightheader{\hspace{-0.8cm}\includegraphics[width=4.5cm]{figs/QE.png}}
In this tutorial we will see how to use the code \prog{cp.x}. We will:
\begin{enumerate}
\item start a BOMD simulation with conjugate gradient minimization of the DFT energy
\item run a Car-Parrinello molecular dynamics simulation (CPMD)
\item run the Nose-Hoover thermostat
\item run a longer microcanonical ensemble simulation
\item view the trajectory and calculate $g(r)$ and mean square displacement from the simulation trajectory
\end{enumerate}

%%%%%%%%%%%%%%%%%%%%%%%%%%%%%%%%%%%%%%%%%%%%%%%%%%%%%%%%%%%%% 
\head{BOMD: small recap}
\begin{itemize}
\item trajectory: time series of position, velocities and in general many quantities that depend on time. It is written on many files by the code
\item Born Oppenheimer approximation: nuclei are classical
\item adiabaticity: electrons are always in their ground state at every timestep
\item Newton equation of motion for the nuclei
\end{itemize}

%%%%%%%%%%%%%%%%%%%%%%%%%%%%%%%%%%%%%%%%%%%%%%%%%%%%%%%%%%%%
\head{Start of the simulation}
We are going to integrate the following equation of motions:
\begin{equation}
	M_I\ddot R_I = -\nabla_{R_I} E_{DFT}(\{R_J\}_{J\in\text{nuclei}})
\end{equation}
That will be discretized according to a Verlet scheme. We have to select all the parameters that are needed to evaulate the forces via the Hellman-Feynman theorem.
That means selecting plane wave cutoff, pseudopotentials, etc.

%%%%%%%%%%%%%%%%%%%%%%%%%%%%%%%%%%%%%%%%%%%%%%%%%%%%%%%%%%%%% 
\head{BOMD cp.x's input file}
\begin{minipage}{0.5\textwidth}
\card{
 \&control \\
    title = 'Water 8 molecules',\\
    calculation = 'cp',\\
    restart\_mode = 'from\_scratch',\\
    ndr = 50,\\
    ndw = 50,\\
    nstep  = 100,\\
    iprint = 10,\\
    isave  = 1000,\\
    tprnfor = .TRUE.,\\
    dt    = 5.0d0,\\
    prefix = 'h2o',\\
    pseudo\_dir='../pseudo',\\
    outdir='./',\\
 /\\
}
\end{minipage}
\begin{minipage}{0.5\textwidth}
\card{
 \&system\\
    ibrav=1, celldm(1)=13.00000\\
    nat = 24,\\
    ntyp = 2,\\
    ecutwfc = 50.0,\\
 /\\
 \&electrons\\
    emass = 50.d0,\\
    emass\_cutoff = 2.5d0,\\
    electron\_dynamics = 'cg',\\
 /\\
 \&ions\\
    ion\_dynamics = 'verlet',\\
    !ion\_velocities = 'zero',\\
    ion\_velocities = 'random',\\
    tempw=600.d0\\
	/\\
}
\end{minipage}


%%%%%%%%%%%%%%%%%%%%%%%%%%%%%%%%%%%%%%%%%%%%%%%%%%%%%%%%%%%%% 
\head{BOMD cp.x's input file}
\begin{minipage}{0.5\textwidth}
\card{
 \&cell\\
    cell\_dynamics = 'none',
 /\\
ATOMIC\_SPECIES\\
 O  16.0d0   O\_ONCV\_PBE-1.2.upf\\
 H  1.0079d0 H\_ONCV\_PBE-1.2.upf\\
ATOMIC\_POSITIONS (bohr)\\
O 0.48E+01 0.37E+01 0.37E+01\\
H 0.40E+01 0.59E+01 0.35E+01\\
.\\
.\\
.\\
}
put here all the atomic positions\\
\end{minipage}
\begin{minipage}{0.5\textwidth}
then, at the end of the input file\\
\card{
AUTOPILOT\\
on\_step=10 : dt = 20.d0\\
on\_step=90 : dt = 5.d0\\
ENDRULES\\
}
This will change the dt parameter during the simulation.
\end{minipage}\\

\hrule The documentation for the AUTOPILOT module can be found at
\url{https://gitlab.com/QEF/q-e/-/blob/develop/CPV/Doc/README.AUTOPILOT}
It is possible to modify some parameters on the fly while the simulation is running by placing a special file called \file{pilot.mb} inside the folder where you are running the simulation

%%%%%%%%%%%%%%%%%%%%%%%%%%%%%%%%%%%%%%%%%%%%%%%%%%%%%%%%%%%%%
\head{BOMD run and tips on initial state}
Ok, now try to run the input file on the cluster!\\
\code{remote\_mpirun cp.x -in cp.water8.1-bomd.in}\\
While it runs you can have a look on how it was created the input file. I took a water molecule, rotate it randomly and then apply other rotation matrixes of some symmetry groups to get a filled cell. A different option could be starting from a crystal, that is usually simpler for structures like say NaCl.

If you want to open the jupyter notebook file, you have to go inside the \code{Day-8/misc} folder, open a terminal and install some python code (if not already done) with \code{pip install jupyter numpy scipy k3d}, then you can open the notebook with \code{~/.local/bin/jupyter notebook} and play with it. Learning python is beyond the scope of this tutorial, but I think is good to know that those tools exists.

%%%%%%%%%%%%%%%%%%%%%%%%%%%%%%%%%%%%%%%%%%%%%%%%%%%%%%%%%%%%%
\head{See what was produced}
\begin{itemize}
  \item download results from the cluster
  \item you see a number of files (everything in Hartree atomic unit):
	  \begin{itemize}
		  \item \code{h2o.cel} file that contains the transposed cell vectors
		  \item \code{h2o.pos} unwrapped positions of the atoms, same atomic order of the input files (!)
		  \item \code{h2o.vel} atomic velocities, same atomic order of the input file (!)
		  \item \code{h2o.evp} thermodynamic data. At first you should look at this
		  \item ...
	  \end{itemize}
\end{itemize}

%%%%%%%%%%%%%%%%%%%%%%%%%%%%%%%%%%%%%%%%%%%%%%%%%%%%%%%%%%%%%
\head{h2o.evp}
First line of file
\code{\#   nfi    time(ps)        ekinc        T\_cell(K)     Tion(K)          etot               enthal               econs               econt          Volume        Pressure(GPa) }
\begin{itemize}
	\item \code{ekinc} $K_{ELECTRONS}$
	\item \code{enthal} $E_{DFT}+PV$ 
	\item \code{etot} $E_{DFT}$ potential energy of the system
	\item \code{econs} $E_{DFT} + K_{NUCLEI}$ this is something that is a constan of motion in the limit where the electronic fictitious mass is zero. It has a physical meaning.
	\item \code{econt} $E_{DFT} + K_{IONS} + K_{ELECTRONS}$ this is a constant of motion of the lagrangian. If the dt is small enough this will be up to a very good precision a costant. It is not a physical quantity, since $K_{ELECTRONS}$ has \emph{nothing} to do with the quantum kinetic energy of the electrons.
\end{itemize}

Note that if the verlet algorithm is not used there is no $K_{ELECTRONS}$, since they don't have a velocity defined

%%%%%%%%%%%%%%%%%%%%%%%%%%%%%%%%%%%%%%%%%%%%%%%%%%%%%%%%%%%%%
\head{See the trajectory}
To see the trajectory we convert the output file to something that is readable from, for example, ovito. In the VM there is a simple python script that you can call for this purpose:

\code{cp2lammpstrj.py h2o -n 24 --minimal --not-ordered  -t 0 1 1 0 1 1 0 1 1 0 1 1 0 1 1 0 1 1 0 1 1 0 1 1 -o h2o\_txtlammps}

then you can open the file \code{h2o\_txtlammps.lammps} with ovito:

\code{ovito h2o\_txtlammps.lammps}

Now you can see the atoms! They didn't move a lot with such a small number of steps...

%%%%%%%%%%%%%%%%%%%%%%%%%%%%%%%%%%%%%%%%%%%%%%%%%%%%%%%%%%%%%
\head{CPMD start}

Now we will modify the input file of the BOMD to do the verlet integration of the Car-Parrinello lagrangian. Few modifications are necessary:
\begin{itemize}
    \item set \verb\calculation = 'restart'\ to start from a previously stopped calculation
    \item set \verb\ndw = 51\ (increase by one the number of the folder where the code will write the restart file)
    \item set \verb\nstep = 1000\ if you want to run for 1000 steps
    \item set \verb\electron_dynamics = 'verlet'\ to set the verlet algorithm to integrate the Car-Parrinello equation of morion
    \item set \verb\ion_velocities = 'default'\ to read the velocity from the specified restart file
    \item remove the \verb\AUTOPILOT\ card
\end{itemize}

%%%%%%%%%%%%%%%%%%%%%%%%%%%%%%%%%%%%%%%%%%%%%%%%%%%%%%%%%%%%%
\head{Output produced }
After syncing again the files, look at \code{h2o.evp}. Now you see that the \code{ekinc} column is not zero! Verify that the constant of motion is conserved with a good approximation. That means that the highest frequency of the system are sampled with a reasonable rate during the integration of the equation. Remember that we have fast oscillating electronic degrees of freedom.

If you look again at the trajectory (after executing again the commands to convert it), you'll see that it's longer. Every time the trajectory data is appendend to the output files \code{h2o.???}

Maybe you noticed the file \code{h2o.for}. This contains the computed force, and is printed only if in the input you have \code{tprnfor = .true.}. Note that there is a factor two between \code{cp.x}'s forces and \code{pw.x}'s one. Do you think that they are more or less the same?

(OPTIONAL) You can, as an experiment, try to pick a timestep, copy and paste atomic positions inside a \code{pw.x} input file and compute the force. Then plot forces CP vs forces PW. How are the ratios distributed?

%%%%%%%%%%%%%%%%%%%%%%%%%%%%%%%%%%%%%%%%%%%%%%%%%%%%%%%%%%%%%
\head{Nose-Hoover thermostat}


\begin{itemize}
    \item set the number of steps to 1000 with \verb\nstep = 1000\
    \item increase by one \verb\ndr\ and \verb\ndw\
    \item set the Nose-Hoover thermostat (namelist \verb\IONS\):
    \begin{itemize}
        \item set nose: \verb\ion_temperature = 'nose'\
        \item temperature: \verb\tempw = 600.d0\
        \item nose frequency: \verb\fnosep = 5.d0\ 
        \item number of thermostat in the NH chain: \verb\nhpcl = 3\ 
    \end{itemize}
\end{itemize}

%%%%%%%%%%%%%%%%%%%%%%%%%%%%%%%%%%%%%%%%%%%%%%%%%%%%%%%%%%%%%
\head{See what happened}

%%%%%%%%%%%%%%%%%%%%%%%%%%%%%%%%%%%%%%%%%%%%%%%%%%%%%%%%%%%%%
\head{CG step}

%%%%%%%%%%%%%%%%%%%%%%%%%%%%%%%%%%%%%%%%%%%%%%%%%%%%%%%%%%%%%
\head{final NVE simulation}

%%%%%%%%%%%%%%%%%%%%%%%%%%%%%%%%%%%%%%%%%%%%%%%%%%%%%%%%%%%%%
\head{look at the trajectory}

%%%%%%%%%%%%%%%%%%%%%%%%%%%%%%%%%%%%%%%%%%%%%%%%%%%%%%%%%%%%%
\head{$g(r)$ plot}

%%%%%%%%%%%%%%%%%%%%%%%%%%%%%%%%%%%%%%%%%%%%%%%%%%%%%%%%%%%%%
\head{mean square displacement plot}


%%%%%%%%%%%%%%%%%%%%%%%%%%%%%%%%%%%%%%%%%%%%%%%%%%%%%%%%%%%%%
\head{force comparison between CP and BO}


\end{document}
